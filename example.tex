%---------------------------------
%   PACKAGES AND THEMES
%---------------------------------

\documentclass[aspectratio=43]{beamer}
\usetheme[
    % --- CUSTOMISATION OPTIONS
    % Change the color theme
    % VALUES : default, dark, I3S, UCA, (Sakura, Classical)
    color=default,
    % ----------------------
    % Default : frametitle at top;  simple : no frametitle
    % VALUES : default, simple
    mode=default,
    % ----------------------
    % Show (opaque / transparent) or hide (false) the footer
    % VALUES : opaque, transparent, false
    footer=false,
    % ----------------------
    % Show the section (and subsection) at the top right of the slide
    % VALUES : true, false
    toprightsection=false,
    % ----------------------
    % Change the display mode of the number of slides on the top right
    % VALUES : fraction, number, none
    numbering=fraction,
    % ----------------------
    % Show a progressbar at top, bottom, or don't show at all
    % Can be replaced by a static separator bar (static)
    % VALUES : top, bottom, static, none
    progressbar=bottom
]{UCA}



%---------------------------------
%   TITLE PAGE
%---------------------------------

\title[Example]{UCABeamerTheme}
\subtitle{Example}

\author[JUNG]{Victor JUNG}


\institute[UCA]{Universit\'{e} C\^{o}te d'Azur, CNRS, I3S, France\\%
  \texttt{victor.jung@univ-cotedazur.fr}}
\date{\today}


%---------------------------------
%   PRESENTATION SLIDES
%---------------------------------

\begin{document}

\maketitle

\begin{frame}[noframetitle]
    \tableofcontents
\end{frame}


% Lists frame
\section{Lists in Beamer}
\begin{frame}{Lists in Beamer}

This is an unordered list:
\begin{itemize}
    \item Item 1
    \item Item 2
    \item Item 3
\end{itemize}

and this is an ordered list:
\begin{enumerate}
    \item Item 1
    \item Item 2
    \item Item 3
\end{enumerate}

\end{frame}


% Blocks frame
\section{Blocks in Beamer}
\begin{frame}[standout]
    Standout frame
\end{frame}


\subsection{Classic}
\begin{frame}{Blocks in Beamer}{Classic}
    \SmallTitle{Different blocks !}\newline

    \begin{block}{Standard Block}
        This is a standard block.
    \end{block}\newline

    \begin{alertblock}{Alert Message}
        This block presents alert message.
    \end{alertblock}\newline

    \begin{exampleblock}{An example of typesetting tool}
        Example: MS Word, \LaTeX{}
    \end{exampleblock}
\end{frame} 

%------------------------------------------------
\subsection{Custom}

\begin{frame}{Blocks in Beamer}{Custom}

    \begin{cblock}[AlertBlockTitle]{Custom Block}
        This is a custom block
    \end{cblock}\newline

    \begin{cblock}[BlockTitle]{Custom Block 2}
        You can use the color you want
    \end{cblock}\newline

    \begin{cblock}[ExampleBlockTitle]{Code}
        \textbackslash begin\{cblock\}[color]\{Title\}\\
            \% ...\\
        \textbackslash end\{cblock\}
    \end{cblock}

\end{frame}


%------------------------------------------------
\section{Theme Customisation}
\subsection{Options}

\begin{frame}{Options (1)}

    \begin{cblock}[violet]{color}
        Change the color theme\\
        \textbf{VALUES :} default, dark, I3S, UCA, Sakura, Classical
    \end{cblock}\nline

    \begin{cblock}[blue]{mode}
        Default : frametitle at top;  simple : no frametitle\\
        \textbf{VALUES :} default, simple
    \end{cblock}\nline

    \begin{cblock}[green]{footer}
        Show (opaque / transparent) or hide (false) the footer\\
        \textbf{VALUES :} opaque, transparent, false
    \end{cblock}

\end{frame}

\begin{frame}{Options (2)}

    \begin{cblock}[yellow]{toprightsection}
        Show the section (and subsection) at the top right of the slide\\
        \textbf{VALUES :} true, false
    \end{cblock}\nline

    \begin{cblock}[orange]{numbering}
        Change the display mode of the number of slides on the top right\\
        \textbf{VALUES :} fraction, number, none
    \end{cblock}\nline

    \begin{cblock}[red]{progressbar}
        Show a progressbar at top, bottom, or don't show at all\\
        Can be replaced by a static separator bar (static)\\
        \textbf{VALUES :} top, bottom, static, none
    \end{cblock}

\end{frame}

%------------------------------------------------
\subsection{Logos}

\begin{frame}{Logos}
    To change the logos for the title page and the footer, you must go to the \InlineCode{./theme/thememacros.sty} file and edit the \textbackslash TitlePageLogo and \textbackslash FooterLogo definitions.\nline 

    \begin{exampleblock}{Already defined logos}

        \includegraphics[height=3em]{\WhiteLogoI3S}
        \hfill
        \includegraphics[height=3em]{\LogoI3S}
        \hfill
        \includegraphics[height=3em]{\WhiteLogoUCA}
        \hfill
        \includegraphics[height=3em]{\LogoUCA}
        \hfill
        \includegraphics[height=3em]{\WhiteLogoInria}
        \hfill
        \includegraphics[height=3em]{\LogoInria}
        \hfill
        \includegraphics[height=3em]{\Logo3IA}

    \end{exampleblock}
\end{frame}

%----------------------------------------------------------------------------------------

\end{document}